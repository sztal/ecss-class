%%% TITLE AND AUTHORS %%%%%%%%%%%%%%%%%%%%%%%%%%%%%%%%%%%%%%%%%%%%%%%%%%%%%%%%%
\title[Final Project]{
    Final Project \\
    \small{and some examples}
}
\author[]{Szymon Talaga and Mikołaj Biesaga} % Your name
\institute[ISS UW]{
    The Robert Zajonc Institute for Social Studies \\ University of Warsaw \\
    \medskip
    \textcolor{blue}{\href{mailto:stalaga@uw.edu.pl}{stalaga@uw.edu.pl}} \\
    \textcolor{blue}{\href{mailto:m.biesaga@uw.edu.pl}{m.biesaga@uw.edu.pl}}
}
\date{\today} % Date, can be changed to a custom date

%%% SLIDES %%%%%%%%%%%%%%%%%%%%%%%%%%%%%%%%%%%%%%%%%%%%%%%%%%%%%%%%%%%%%%%%%%%%
\frame{\titlepage}

\section{Final project}

\begin{frame}{Possible forms of the project}
\begin{enumerate}
    \item \textbf{Research project proposal.}
    Short proposal for a research project using some of
    the methods we discussed in the class. Details are discussed
    in \textit{Structure of the proposal} section.
    \item \textbf{Simple research project and report.}
    Simple data-driven project in which you gather some data from an external
    web-based source (i.e. from an API or through webscraping), process it
    and use it to answer a simple research question. Details are discussed in
    \textit{Structure of the project} section.
\end{enumerate}
\end{frame}

\begin{frame}{Structure of the proposal | general information}
\begin{itemize}
    \item \textbf{Introduction and problem statement.}
    This section should explain what is the project about in general
    and why the given problem matters. Significance of the problem may
    be justified both in terms of its theoretical/scientific
    or societal/business relevance.
    \item \textbf{Research question.}
    What is the specific research question you will try to answer in the
    project. It can be either formulated as a strict hypothesis or as an
    exploratory question (i.e. not assuming any particular effect or mechanism).
    \item \textbf{Research methods.}
    This section should discuss, at a general level, what data do you plan
    to use and how will analyze it. In particular, you should clearly explain
    why the data and methods you chose will allow you to answer your research
    question.
\end{itemize}
\end{frame}

\begin{frame}{Structure of the proposal | research methods}
\begin{itemize}
    \item \textbf{Data collection.} What is the data you are planning to use?
    Where is it stored and how do you plan to obtain/extract it?
    Are there any possible obstacles and if there are what can be done to
    minimize the risk of a failure?
    \item \textbf{Data preprocessing and storage.}
    How will you preprocess data and in what format will store it for later
    use? For instance, when scraping you may decide to data gathered from
    every individual website to a JSON object and store all data in
    \texttt{.jl} file (JSON lines).
    \item \textbf{Analytic methods.}
    What analytic methods will you apply to your data in order to answer
    your question? This may include some methods that we discussed in the class
    (i.e. sentiment analysis or some other natural language processing methods),
    but you should also use your general statistical knowledge to decide
    how you want to model your final data.
\end{itemize}
\end{frame}

\begin{frame}{Structure of the project}
\begin{itemize}
    \item The project should consist of two main items:
    \begin{enumerate}
        \item \textbf{Notebook} with code and description of the project.
        More on that below.
        \item File with the \textbf{raw data} as was extracted from
        the external data source (e.g. API or webpage).
    \end{enumerate}
    \item The notebook should include short introduction describing the
    problem and why it is significant.
    \item The specific research question should be introduced.
    \item Data source should be discussed. In particular in relation to
    the research question.
    \item Data analysis methods should be discusses, also in relation to the
    research question.
    \item Code and text should be mixed in the notebook in a way that
    facilitates understanding of the adopted research methods and obtained
    results.
\end{itemize}
\end{frame}

\begin{frame}{Things to remember}
\begin{itemize}
    \item Make sure you will be able to gather data legally.
    For instance, in want to scrape some websites you should try
    to determine whether they allow scraping. You can check this by
    looking at so-called \texttt{robots.txt} of a website.
    It can be found at \texttt{<url>/robots.txt}. For instance,
    you can see Facebook configuration at \texttt{facebook.com/robots.txt}.
    \item Be kind for your data source. If there is a platform that exposes
    an API you should use it instead of scraping data directly from its
    webpage.
    \item Think about proper format for storing data. If you plan to use
    your data to compute a statistical model in SPSS maybe JSON lines are not
    the best and you should consider storing your data as a simple CSV file?
\end{itemize}
\end{frame}

% \section[Example]{Example of a research proposal}

% \begin{frame}{}
% \begin{itemize}
%     \item 4
% \end{itemize}
% \end{frame}

%%% LITERATURE %%%%%%%%%%%%%%%%%%%%%%%%%%%%%%%%%%%%%%%%%%%%%%%%%%%%%%%%%%%%%%%%
% \begin{frame}[allowframebreaks]
% \frametitle{Literature}
% \nocite{*}
% \AtNextBibliography{\footnotesize}
% \printbibliography[title=Literature]
% \end{frame}
